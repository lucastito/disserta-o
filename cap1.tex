\chapter{Introdução}

Em fevereiro de 2015, as autoridades brasileiras começaram a investigar um surto de erupções cutâneas que afetavam seis (6) estados na região nordeste do país e constatou-se que um vírus desconhecido até então nas Américas começou a circular: o vírus Zika (ZIKV). A presença do Zika e as graves anomalias congênitas causadas por ele em bebês, em especial a microcefalia, apresentaram-se como uma situação inusitada que mereceu, inclusive, declarações de Emergência em Saúde Pública de importância nacional e internacional, tornando-se uma ameaça de carater mundial.
De acordo com a Organização Pan-Americana da Saúde (OPS), 20 países nas Américas  já relataram a detecção de ZIKV em seus territórios. No cenário mundial, mais de 73 países nos cinco (5) continentes relatam a infecção por ZIKV.
A bioinformática de um modo geral visa o apoio à gerência, tratamento, análise, integração, e interpretação da informação dos dados biológicos de diversos experimentos em larga escala. Em Ciência de Dados, compreender, processar, e inferir conhecimento sobre experimentos complexos e específicos que consomem, processam e geram grande volume de dados são desafios ainda em aberto para a comunidade científica em geral. Os dados científicos, clínicos, biológicos e epidemiológicos relacionados só a essas arboviroses se comportam como dados biológicos big data. Por sua vez, estima-se que atinja os 44 zettabytes em 2020. Mais especificamente, dados genômicos são dados complexos, heterogêneos e multilocalizados. Tecnologias de gerência, armazenamento e execução em larga escala são necessárias para o processamento, tratamento e análise desses dados.
O Ministério da Saúde centraliza dados e metadados em data warehouses como o Sistema Gerenciador de Ambiente Laboratorial GAL \ref{http://gal.datasus.gov.br/}{GAL}.
O GAL é um banco de dados que contém as informações de pacientes de todo o Brasil com suspeita de estarem infectados com alguma das doenças emergentes e reemergentes mencionadas, cujas amostras clínicas são enviadas para os laboratórios que compõem o SISLAB (Sistema de Laboratórios de Saúde Pública do SUS). Esse banco centraliza dados epidemiológicos, clínicos e diagnósticos provendo características de segurança, confidencialidade, ética e propriedade intelectual aos dados.
Com o advento das tecnologias de sequenciamento e técnicas de bioinformática, dados genômicos precisam ser avaliados de forma eficiente e rápida, especialmente no caso de pacientes infectados. Nesse projeto, um dos objetivos é interconectar os dados armazenados e disponibilizados do Gal com dados genômicos resultantes das análises de Laboratórios de Referência. Uma vez integrados, esses dados podem ser tratados e processados para posteriores análises de bioinformática usando técnicas e ambientes de processamento computacional especializados.