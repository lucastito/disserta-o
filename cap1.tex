\chapter{Introdução}

Em fevereiro de 2015, as autoridades brasileiras começaram a investigar um surto de erupções cutâneas que afetavam 6 estados na região nordeste do país e constatou-se que um vírus desconhecido até então nas Américas começou a circular: o vírus Zika (ZIKV). A presença do Zika e as graves anomalias congênitas causadas por ele em bebês, em especial a microcefalia, apresentaram-se como uma situação inusitada que mereceu, inclusive, declarações de Emergência em Saúde Pública de importância nacional e internacional, tornando-se uma ameaça de caráter mundial.
De acordo com a Organização Pan-Americana da Saúde (OPS), 48 países nas Américas  já relataram a detecção desse arbovirus em seus territórios \cite{ops}, enquanto que no cenário mundial, segundo a Organização Mundial de Saúde (OMS) 86 países nos 5 continentes relataram a infecção \cite{who}.
Os dados científicos, clínicos, biológicos e epidemiológicos relacionados as arboviroses já se comportam como dados biológicos big data. Por sua vez, estima-se que atinja os 44 zettabytes em 2020.
No Brasil, o Ministério da Saúde centraliza dados e metadados em data warehouses como o Sistema Gerenciador de Ambiente Laboratorial GAL \ref{http://gal.datasus.gov.br/}{GAL}.
O GAL é um banco de dados que contém as informações de pacientes de todo o Brasil com suspeita de estarem infectados com alguma das doenças emergentes e reemergentes mencionadas, cujas amostras clínicas são enviadas para os laboratórios que compõem o SISLAB (Sistema de Laboratórios de Saúde Pública do SUS). Esse banco centraliza dados epidemiológicos, clínicos e diagnósticos provendo características de segurança, confidencialidade, ética e propriedade intelectual aos dados.
A gerência, tratamento, análise, integração, e interpretação da informação dos dados biológicos são etapas necessárias para a ciência e a gestão dos recursos utilizados para a manutenção da saúde pública. Este trabalho tem como objetivo aplicar técnicas de Data Science no contexto biológico para a descoberta de conhecimento de dados que possam ser utilizados pelo Ministério da Saúde e/ou laboratórios que pesquisem essas doenças.
