\chapter{Introdução}

Na última década, situações inusitadas de proporções mundiais sobre surtos de doenças emergentes e reemergentes transmitidas por mosquitos Aedes como a febre amarela, dengue, Zika e chikungunya acionaram declarações de emergência em saúde pública de importância nacional e internacional, por parte do governo brasileiro e da Organização Mundial da Saúde (OMS). A partir dessas declarações, houve uma intensa mobilização científica mundial acerca do entendimento da epidemia causada pelo ZIKV e seus efeitos e com isso, aumentou rápida e significativamente a produção científica envolvendo a epidemiologia, clínica e seus mecanismos de infecção, embora muito ainda seja desconhecido. Os dados científicos, clínicos, biológicos e epidemiológicos relacionados só a essas arboviroses se comportam como dados biológicos big data. O processamento e análise desse grande volume de dados biológicos (estima-se que atinja os 44 zettabytes em 2020) complexos, heterogêneos e distribuídos torna-se um desafio para as Ciências Biológicas, de Dados, da Bioinformática e Computacional.
Dados genômicos são dados complexos, heterogêneos e multilocalizados. Tecnologias de gerência, armazenamento e execução em larga escala são necessárias para o processamento, tratamento e análise de dados. Torna-se importante aplicar tecnologias de pesquisas multidisciplinares que possam levar à descoberta de compostos como vacinas para ZIKV, de interesse social e de saúde no Brasil. 
No Brasil dados e metadados relacionados a doenças genômicas infecciosas como é o caso da febre amarela, chikungunha, dengue e Zika, entre outros, encontram-se centralizados em Laboratórios de Referência como a FIOCRUZ no Brasil. O Ministério da Saúde centraliza dados e metadados epidemiológicos, clínicos, diagnósticos de doenças em data warehouses como o Sistema Gerenciador de Ambiente Laboratorial GAL (http://gal.datasus.gov.br/) do Ministério da Saúde. O GAL é um banco de dados que centraliza dados epidemiológicos, clínicos e diagnósticos de pacientes e amostras de todo o Brasil, provendo características de segurança, confidencialidade, ética e propriedade intelectual aos dados. Esse banco é composto por dados de pacientes suspeitos de doenças cujas amostras clínicas são enviadas para os laboratórios que compõem o SISLAB (Sistema de Laboratórios de Saúde Pública do SUS). Com o advento das tecnologias de sequenciamento e técnicas de bioinformática, dados genômicos precisam ser avaliados de forma eficiente e rápida, especialmente no caso de pacientes infectados. Nesse projeto, um dos objetivos é interconectar os dados armazenados e disponibilizados do Gal com dados genômicos resultantes das análises de Laboratórios de Referência. Uma vez integrados, esses dados podem ser tratados e processados para posteriores análises de bioinformática usando técnicas e ambiente de processamento computacional especializado. 
e-Ciência pode apoiar a integração e organização dos dados de uma forma científica, por meio de tecnologias de ciência de dados, banco de dados, computação de alto desempenho e bioinformática e biológica computacional. Com a natureza de alta infectividade dessas doenças, este trabalho se propõe a aplicar experimentos de filodinâmica e filogenômica aos metadados genômicos, pelo que é requerido a maior quantidade de metadados e informações para enriquecer a análise. 
-------
existem 17 NTD  endêmicas em 149 países que afetam a mais de 1 bilhão de pessoas (1/7 da população mundial). 
Em fevereiro de 2015, as autoridades brasileiras começaram a investigar um surto de erupções cutâneas que afetavam seis (6) estados na região nordeste do país e constatou-se que um vírus desconhecido até então nas Américas começou a circular: o vírus Zika (ZIKV). A presença do Zika e as graves anomalias congênitas causadas por ele em bebês, em especial a microcefalia, apresentaram-se como uma situação inusitada que mereceu, inclusive, declarações de Emergência em Saúde Pública de importância nacional e internacional, por parte do governo brasileiro e da OMS, respectivamente, diante da gravidade da epidemia que surgiu no Brasil e que logo passou a representar ameaça em diferentes países do mundo . A demonstração de que o vírus Zika leva a microcefalia foi reconhecida pela OMS em 2016 e escreveu um novo capítulo da história da medicina. De acordo com a Organização Pan-Americana da Saúde (OPS), 20 países nas Américas  já relataram a detecção de ZIKV no seu território. No entanto, a grande epidemia de ZIKV que aconteceu no Brasil teve seu início possivelmente em 2014 e se espalhou de forma explosiva ao longo do eixo das Américas, tomando proporções mundiais em 2016. Atualmente, mais de 73 países nos cinco (5) continentes relatam a infecção por ZIKV .
Três (3) grandes discussões vieram à tona durante os primeiros meses do aumento de casos de microcefalia: controle da natalidade, aborto e, com menos destaque, as condições precárias de saneamento do País. Este um problema associado não apenas ao Zika, mas a outras doenças ligadas ao Aedes aegypti, dengue, febre amarela e chikungunya.

A partir das declarações de emergências nacional e internacional, houve uma intensa mobilização científica mundial acerca do entendimento da epidemia causada pelo ZIKV e seus efeitos. Em 2014, buscas no PubMed  usando a palavra-chave “microcefalia” geravam poucos artigos, ainda quando refinada a pesquisa, cruzando as palavras-chaves “microcefalia”, “dengue”, “chikungunha” e “Zika”, os resultados eram ainda menores, o que demonstrou como a ligação entre a malformação e as arboviroses era algo totalmente novo. Até 15 de junho de 2016 foram encontrados no PubMed cerca de 830 artigos (686, só em 2016), em revistas nacionais e internacionais, algumas consideradas as mais importantes, muito concorridas e de enorme dificuldade para se ter um artigo aceito para publicação. Enquanto que em 2014 pudemos encontrar 27 artigos relacionados a essa temática e, em 2015, foram 42 artigos, em 2016 (até setembro) tivemos 1.277 publicações, numa média de 270 trabalhos publicados a cada semana. Somente no ano de 2016, foram publicados mais de 56 artigos relacionados ao ZIKV nas revistas internacionais mais importantes do ponto de vista da divulgação do conhecimento científico. Quando se analisa a publicação considerando a tríplice epidemia DENV, ZIKV e CKIKV foram mais de 120 artigos publicados. Essas informações demonstram a natureza heterogênea e dispersa dos metadados, o que pode inviabilizar os experimentos a posteriori. Um data warehouse como via de centralização e extração desses metadados genômicos, epidemiológicos e textuais (do Pubmed e NCBI) de DTN e ZIKV pretende ser desenvolvido nesse projeto.
----
A bioinformática de um modo geral visa o apoio à gerência, tratamento, análise, integração, e interpretação da informação dos dados biológicos de diversos experimentos em larga escala. No entanto, devido ao incremento exponencial e complexidade desses dados, incrementou-se também a necessidade de tecnologias e recursos humanos especializados para o desenvolvimento de algoritmos que apoiem a interpretação efetiva desses dados biológicos. Em Ciência de Dados, compreender, processar, e inferir conhecimento sobre experimentos complexos e específicos como aqueles da bioinformática que consomem, processam e geram grande volume de dados biológicos (biological big data) são desafios ainda em aberto para a comunidade científica em geral, incluindo as Ciências Exatas e Ciências Biológicas.
Recentes progressos usando tecnologias eficientes e escaláveis em mineração de dados e aprendizado de máquinas têm se tornado interessantes para o apoio à análise de conhecimento, informação e interação com padrões em grandes bancos de dados. Esses métodos de classificação eficiente exploram o agrupamento, análises de valores e resultados atípicos, frequência, métodos de análise de padrões estruturados e ferramentas de análise de dados temporal/espacial e visualização. A mineração e aprendizado no apoio a análise de metadados (dados de proveniência de domínio específico) têm sido de grande ajuda no reconhecimento de padrões e interpretação analítica de dados e já são aplicados na bioinformática em processos que se referem à procura de motivos ou padrões nas sequências de DNA e proteínas; descoberta de mecanismos relacionados a doenças; levantamento de regras de agrupamento de sequências; procura de padrões de táxons em clados frequentes em árvores filogenéticas ou na interpretação de dados de microarranjos. 
Enquanto à análise de dados científicos, nesse projeto focado nas arboviroses e ZIKV, é importante integrar dados de diferentes fontes (i.e., metagenômicos, genômicos, proteômicos, transcriptômicos, etc) com bancos de dados clínicos de pacientes e, no caso da distribuição de uma epidemia, com dados geográficos, ambientais e epidemiológicos. Recentes progressos na análise desse grande volume de dados requerem técnicas que apoiem na integração dos metadados e dados de proveniência de um experimento de bioinformática. Técnicas clássicas de mineração e aprendizado evoluem à medida que a complexidade, heterogeneidade, volume, e dispersão geográfica dos dados (característica “big data”) é incrementado. Essas técnicas precisam ser melhor acopladas às técnicas de processamento massivo paralelo como PAD e SGWfC, o que têm levado ao desenvolvimento de numerosos métodos eficientes/escaláveis para a gerência e análise deste tipo de dados. Utilizando efetivamente conhecimentos adquiridos de áreas exatas como mineração, aprendizado, computação paralela, workflows científicos, banco de dados escaláveis, gerência e segurança de grandes volumes de dados, é possível oferecer um apoio efetivo às áreas aplicadas como a bioinformática. A integração efetiva dessa pesquisa multidisciplinar poderá apoiar as análises do grande volume de dados biológicos gerados pelo nosso grupo de pesquisa no LNCC e colaboradores no Brasil e no exterior, e ao mesmo tempo alavancar pesquisas para o Brasil se posicionar à par dos descobrimentos científicos mundiais.
O desafio em questão é a melhor maneira de processar, gerenciar e analisar os dados científicos em ambientes de PAD e torná-la em uma metodologia exitosa. Os cientistas precisam de um grau de abstração, a fim de integrar eficientemente experimentos de bioinformática às várias tecnologias computacionais como SGWfC, SGBD, aplicações aceleradas por GPU (graphics processing unit), mineração de dados, aprendizagem de máquina, sistemas de segurança e arquiteturas de PAD especializadas para a análises de grande volume de dados como arquiteturas MapReduce (Hadoop), gráficas de tolerância a falhas (GraphLab) ou gráficas de streaming (Spark) em clusters de supercomputadores. 
Por tanto, esse projeto visa absorver e transformar as pesquisas desenvolvidas neste projeto em produtos a serem disponibilizados no apoio a análises dos dados de Sistemas como o Gal, a modo de um arcabouço computacional especializado baseado em proveniência de dados de experimentos científicos de bioinformática executados em ambientes distribuídos e paralelos. O desenvolvimento desse arcabouço será guiado pelos aspectos da organização e gerência de dados em experimentos científicos em larga escala. Como estudos de caso serão explorados experimentos científicos de genômica com o intuito de desenvolver e implementar a arquitetura que apoie a mineração dos dados, metadados e proveniência dos resultados relacionados a esses experimentos. Mais especificamente este projeto visa: (i) aplicar técnicas de mineração e aprendizado (e.g. busca de padrões e informação de dados) na forma de workflows científicos encima dos dados de desempenho, proveniência, e do domínio específico em áreas específicas da bioinformática, (ii) desenvolver um arcabouço computacional baseado em SGWfC, SGBD e WebService que integre esses workflows científicos e a extensão do Gal (iii) acoplar eficientemente esse arcabouço em ambientes de PAD para executar e analisar esses workflows de forma paralela e distribuída (iv) analisar os dados biológicos, estrutura-los em data warehouse e aplicar tecnologias de aprendizado de máquinas para procurar padrões que levem a levantar hipóteses sobre a evolução, filogenômica, filodinâmica e epidemiologia de arboviroses e ZIKV, com o intuito de esclarecer dúvidas da vida evolutiva dessas espécies que levem a avanços biotecnológicos no desenvolvimento de kits e vacinas. Cabe ressaltar que as pesquisas sobre as tecnologias levantadas e implementadas poderão ser extrapoladas a outras pesquisas nos alvos de interesse clínico (câncer, doenças genéticas) ou biotecnológico (biomarcadores, vacinas) no Brasil.
