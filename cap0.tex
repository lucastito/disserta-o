% --- -----------------------------------------------------------------
% --- Elementos usados na Capa e na Folha de Rosto.
% --- EXPRESSOES ENTRE <> DEVERAO SER COMPLETADAS COM A INFORMACAO ESPECIFICA DO TRABALHO
% --- E OS SIMBOLOS <> DEVEM SER RETIRADOS 
% --- -----------------------------------------------------------------
\autor{LUCAS DE SOUZA TITO} % deve ser escrito em maiusculo

\titulo{Apoio Computacional na Predição Epidemiológica e Filodinâmica de Doenças Infecciosas Emergentes}

\instituicao{UNIVERSIDADE FEDERAL FLUMINENSE}

\orientador{DANIEL CARDOSO MORAES DE OLIVEIRA}

\coorientador{KARY ANN DEL CARMEN OCAÑA GAUTHEROT} % se nao existir co-orientador apague essa linha

\local{NITER\'{O}I}

\data{2019} % ano da defesa

\comentario{Dissertação de Mestrado apresentada ao Programa de P\'{o}s-Gradua\c{c}\~{a}o em Computa\c{c}\~{a}o da \mbox{Universidade} Federal Fluminense como requisito parcial para a obten\c{c}\~{a}o do Grau de \mbox{Mestre em Computa\c{c}\~{a}o}. \'{A}rea de concentra\c{c}\~{a}o: \mbox{Engenharia de Sistemas e Informação}} %preencha com a sua area de concentracao


% --- -----------------------------------------------------------------
% --- Capa. (Capa externa, aquela com as letrinhas douradas)(Obrigatorio)
% --- ----------------------------------------------------------------
\capa

% --- -----------------------------------------------------------------
% --- Folha de rosto. (Obrigatorio)
% --- ----------------------------------------------------------------
\folhaderosto


\pagestyle{ruledheader}
\setcounter{page}{1}
\pagenumbering{roman}

% --- -----------------------------------------------------------------
% --- Termo de aprovacao. (Obrigatorio)
% --- ----------------------------------------------------------------
\cleardoublepage
\thispagestyle{empty}

\vspace{-60mm}

\begin{center}
   {\large LUCAS DE SOUZA TITO}\\
   \vspace{7mm}

Apoio Computacional na Predição Epidemiológica e Filodinâmica de Doenças Infecciosas Emergentes    \\
  \vspace{10mm}
\end{center}

\noindent
\begin{flushright}
\begin{minipage}[t]{8cm}

Dissertação de Mestrado apresentada ao Programa de P\'{o}s-Gradua\c{c}\~{a}o em Computa\c{c}\~{a}o da Universidade Federal Fluminense como requisito parcial para a obten\c{c}\~{a}o do \mbox{Grau} de Mestre em Computa\c{c}\~{a}o. \'{A}rea de concentra\c{c}\~{a}o: \mbox{Engenharia de Sistemas e Informação} %preencha com a sua area de concentracao

\end{minipage}
\end{flushright}
\vspace{1.0 cm}
\noindent
Aprovada em <MES> de <ANO>. \\
\begin{flushright}
  \parbox{11cm}
  {
  \begin{center}
  BANCA EXAMINADORA \\
  \vspace{6mm}
  \rule{11cm}{.1mm} \\
    Prof. Daniel Cardoso Moraes de Oliveira - Orientador, IC/UFF \\
    \vspace{6mm}
  \rule{11cm}{.1mm} \\
    Kary Ann del Carmen Ocana Gautherot, LNCC\\
    \vspace{6mm}
  \rule{11cm}{.1mm} \\
    Prof. <NOME DO AVALIADOR>, <INSTITUI\c{C}\~AO>\\
  \vspace{4mm}
  \rule{11cm}{.1mm} \\
    Prof. <NOME DO AVALIADOR>, <INSTITUI\c{C}\~AO>\\
    \vspace{6mm}
  \rule{11cm}{.1mm} \\
    Prof. <NOME DO AVALIADOR>, <INSTITUI\c{C}\~AO>\\
  \vspace{6mm}
  \end{center}
  }
\end{flushright}
\begin{center}
  \vspace{4mm}
  Niter\'{o}i \\
  %\vspace{6mm}
  2019

\end{center}

% --- -----------------------------------------------------------------
% --- Dedicatoria.(Opcional)
% --- -----------------------------------------------------------------
\cleardoublepage
\thispagestyle{empty}
\vspace*{200mm}

\begin{flushright}
{\em 
Dedicatória(s): Dedico este trabalho a Deus, minha mãe Patrícia Andréa e ao meu irmão Caio Vinícius que são os pilares da minha vida e cujo amor funcionou como força motriz na realização e conclusão desta dissertação.
}
\end{flushright}
\newpage


% --- -----------------------------------------------------------------
% --- Agradecimentos.(Opcional)
% --- -----------------------------------------------------------------
\pretextualchapter{Agradecimentos}
\hspace{5mm}
Gostaria de agradecer aos meus melhores amigos: Marcelo D'Almeida, Kelly Tavares, Nathan Gerhard, Guilherme Alves, Victor Olimpio, Sandra Fratane, Felipe Santiago, Alexandre Estebanez, Felipe Lugão Eccard, Rodrigo Rodovalho e André Alvarado por me apoiarem e ajudarem a resolver problemas em código, debugar, configurar ambiente, revisar este trabalho, me ouvirem reclamar da vida ou me obrigarem a sair pra relaxar e não perder a cabeça (não necessariamente nessa ordem); ao melhor professor do IC/UFF, meu orientador Daniel de Oliveira, por todos os conselhos, toda a ajuda e principalmente, por toda a paciência comigo quando eu o perturbava; e minha gestora Paula Mian por toda a compreensão e flexibilidade nos meus horários tornando viável a continuidade do mestrado em paralelo a minhas atividades na Muxi Tecnologia S.A o qual sou grato pelo tempo de empresa e todo o aprendizado adquirido; ao meu avô Florentino mangueira, por compartilhar comigo suas histórias, suas experiências de vida e por servir como um grande exemplo de perseverança; ao Roberto Carlos Melo, por ajudar na construção do meu caráter e me auxiliar em diferentes momentos do meu cotidiano; a todos os meus professores e à todas as minhas professoras, do jardim até a pós-graduação, que passaram parte dos seus conhecimentos para mim. Também aos meus colegas e familiares que sempre foram presentes apesar da distância. Além daqueles que encontrei diariamente ao decorrer da minha vida e cujo nome não sei ou esqueci (faxineiros, cobradores de ônibus, vendedores ambulantes, porteiros e demais), pessoas que alegraram meu dia, sendo educados e respeitosos, oferecendo alguma ajuda por mais singela que pareça.

% --- -----------------------------------------------------------------
% --- Resumo em portugues.(Obrigatorio)
% --- -----------------------------------------------------------------
\begin{resumo}

Este trabalho apresenta o uso de técnicas e ferramentas de data science e e-science em tarefas de pré-processamento, integração, consulta e visualização dos dados como apoio computacional para a predição epidemiológica e filodinâmica de doenças infecciosas emergentes como Zika. A principal base de dados utilizada foi extraída do Sistema Gerenciador de Ambiente Laboratorial (GAL), que quando integrada a outras informaçães permitiu a obtenção de resultados interessantes, dentre eles uma série temporal das doenças por região.

{\hspace{-8mm} \bf{Palavras-chave}}: Palavras representativas do conteúdo do trabalho, isto é, palavras-chave e/ou descritores, conforme a ABNT NBR 6028 (ABNT, 2005).

\end{resumo}

% --- -----------------------------------------------------------------
% --- Resumo em lingua estrangeira.(Obrigatorio)
% --- -----------------------------------------------------------------
\begin{abstract}

This work introduces the use of data science and e-science tools and techniques in preprocessing, integrating, querying and visualizing data as computational support for the epidemiological and philodynamic prediction of emerging infectious diseases such as Zika. The main database has been extracted from the Laboratory Environment Management System (GAL), which has been integrated with other information, allowing to obtain interesting results, among them a time series of diseases by region.

{\hspace{-8mm} \bf{Keywords}}: Palavras representativas do conteúdo do trabalho, isto é, palavras-chave e/ou descritores, na língua (ABNT, 2005).

\end{abstract}

% --- -----------------------------------------------------------------
% --- Lista de figuras.(Opcional)
% --- -----------------------------------------------------------------
%\cleardoublepage
\listoffigures


% --- -----------------------------------------------------------------
% --- Lista de tabelas.(Opcional)
% --- -----------------------------------------------------------------
\cleardoublepage
%\label{pag:last_page_introduction}
\listoftables
\cleardoublepage

% --- -----------------------------------------------------------------
% --- Lista de abreviatura.(Opcional)
%Elemento opcional, que consiste na relacao alfabetica das abreviaturas e siglas utilizadas no texto, seguidas das %palavras ou expressoes correspondentes grafadas por extenso. Recomenda-se a elaboracao de lista propria para cada %tipo (ABNT, 2005).
% --- ----------------------------------------------------------------
\cleardoublepage
\pretextualchapter{Lista de Abreviaturas e Siglas}
\begin{tabular}{lcl}
<ABREVIATURA> & : & <SIGNIFICADO>;\\
<ABREVIATURA> & : & <SIGNIFICADO>;\\
<ABREVIATURA> & : & <SIGNIFICADO>;\\
\end{tabular}
% --- -----------------------------------------------------------------
% --- Sumario.(Obrigatorio)
% --- -----------------------------------------------------------------
\pagestyle{ruledheader}
\tableofcontents


