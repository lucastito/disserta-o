\chapter{Metodologia}

b.	Desenvolver um arcabouço computacional que integre SGWfC, SGBD e WebService aos workflows científicos de bioinformática. Usar ambientes de PAD para executar/analisar esses workflows e as LE no arcabouço.
c.	Importar os dados (coluna) do GAL para SQL.
d.	Estabelecer o relacionamento, transformação e limpeza dos dados.
e.	Levantar as consultas requeridas pelos especialistas e.g., id do paciente, colunas que precisam de limpeza (dados humanos, animais), extração de pacientes por estado, tratamento e transformação de colunas com dados meteorológicos (Tabela 1).
f.	Integrar os dados do Gal com dados genômicos;
g.	Adequar os dados genômicos do e.g., Zika para as análises de filodinâmica: informações para a filodinâmica do ZIKV.
h.	Modelagem dos workflows de filodinâmica, coevolução e redes metabólicas de interação proteína-proteína. A filogenômica considera a representação de um processo evolutivo e podem incluir várias estimações evolutivas relacionadas a datas de divergência, padrões de diversificação, taxonomia e filodinâmica. A filodinâmica visa inferir as taxas de transmissão e disseminação na evolução de organismos patógenos e o seu cálculo dependente dos dados genômicos serem heterogêneos e suficientes. 
Tabela 1. Consultas SQL propostas
Tipo de SQL		Quais são os ids de todos os pacientes (cruzar por nome ou identificação)
	Estatística por estados brasileiros
	Limpar para obter só dados de humanos
	Quantos negativos/positivos para os testes clínicos e bioquímicos
	Qual a origem dos dados genômicos? (cruzar com as tabelas do Gal)
	Extrair os dados genômicos: dados genômicos completos, parciais
	Realizar análises de bioinformáticas baseada em dados de proveniência
