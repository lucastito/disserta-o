\chapter{Abordagem Proposta}

A abordagem proposta seguiu as seguintes etapas:
I.	Importar os dados do GAL.
II.	Estabelecer o relacionamento, transformação e limpeza dos dados.
III.	Integrar os dados do Gal com dados genômicos.
IV.	Levantar as consultas requeridas pelos especialistas.
V.	exibir o resultado das consultas para o pesquisador por meio da visualização de dados.

%por definição de preprocessamento

Como mencionado no capítulo anterior, para este trabalho, as fontes de dados extraídas foram:
1 planílha do excel exportada pelo GAL contendo duas folhas. A primeira com dados de humanos e a segunda com dados de macacos. Dessas, somente a primeira foi utilizada.
2 base de dados genômicos, extraída do \cite{federhen2011ncbi}


As tarefas de manipulação dos dados na etapa inicial de pré-processamento dos dados foi feita na linguagem python por meio da biblioteca \ref{https://pandas.pydata.org/}{Pandas} e o pacote \ref{http://www.numpy.org/}{NumPy}, que juntas são atualmente algumas das ferramentas mais conhecidas e poderosas em data science.

Record Linkage

A biblioteca \ref{https://recordlinkage.readthedocs.io/en/latest/index.html}{Record Linkage Toolkit} foi utilizada para o processo de deduplicação na base extraída do GAL. Juntamente com o Pandas e o NumPy, possibilitou a criação de uma base limpa e consistente para a etapa de integração descrita no capítulo seguinte.

O Apache Drill é um sistema distribuído gratuito e de código aberto, para análise ad-hoc interativa de conjuntos de dados de grande escala. Projetado para lidar com até
petabytes de dados espalhados por milhares de servidores, seu objetivo é responder a consultas ad-hoc em uma latência baixa.
A arquitetura do drill possui basicamente 3 camadas: usuário, processamento e fonte de dados.
A camada de usuário provê acesso aos dados por meio de interfaces (linha de comando, REST, API e drivers JDBC/ODBC). A camada de processamento permite plugar ou extender linguagens de consulta. Na camada dos dados, configura-se  o acesso as fontes de dado, local e/ou clusterizada. Essas fontes podem ser estruturadas, semiestruturadas ou não estruturadas. \cite{hausenblas2013apache}.