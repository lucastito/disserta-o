\chapter{Pré-Processamento}

%por definição de preprocessamento

Como mencionado no capítulo anterior, para este trabalho, as fontes de dados extraídas foram:
1 planílha do excel exportada pelo GAL contendo duas folhas. A primeira com dados de humanos e a segunda com dados de macacos. Dessas, somente a primeira foi utilizada.
2 base de dados genômicos, extraída do \cite{federhen2011ncbi}


As tarefas de manipulação dos dados na etapa inicial de pré-processamento dos dados foi feita na linguagem python por meio da biblioteca \ref{https://pandas.pydata.org/}{Pandas} e o pacote \ref{http://www.numpy.org/}{NumPy}, que juntas são atualmente algumas das ferramentas mais conhecidas e poderosas em data science.

Record Linkage

O termo record linkage é usado para indicar o procedimento de reunir informações de dois ou mais registros que se acredita pertencerem à mesma entidade.
Esse processo é usado para vincular dados de várias fontes de dados ou para encontrar duplicatas em uma única fonte de dados.
Na ciência da computação, record linkage também é conhecido como data matching ou deduplication(no caso de registros duplicados de pesquisa em um único arquivo).
Em record linkage, os atributos da entidade (armazenados em um registro) são usados para vincular dois ou mais registros.
Atributos podem ser identificadores exclusivos da entidade como o código da pessoa física (CPF) ou o número do registro geral (RG), mas também existem atributos como nome, data de nascimento, modelo do carro e endereço que podem ser utilizados.
O procedimento de record linkage pode ser representado como um fluxo de trabalho com as seguintes etapas: limpeza, indexação, comparação, classificação e avaliação. Se necessário, os pares de registros classificados fluem de volta para melhorar o passo anterior \cite{christen2012data}.
A biblioteca \ref{https://recordlinkage.readthedocs.io/en/latest/index.html}{Record Linkage Toolkit} foi utilizada para o processo de deduplicação na base extraída do GAL. Juntamente com o Pandas e o NumPy, possibilitou a criação de uma base limpa e consistente para a etapa de integração descrita no capítulo seguinte.