\chapter{Consultas}
Tipo de SQL		Quais são os ids de todos os pacientes (cruzar por nome ou identificação)
	Estatística por estados brasileiros
	Limpar para obter só dados de humanos
	Quantos negativos/positivos para os testes clínicos e bioquímicos
	Qual a origem dos dados genômicos? (cruzar com as tabelas do Gal)
	Extrair os dados genômicos: dados genômicos completos, parciais
	Realizar análises de bioinformáticas baseada em dados de proveniência


- Ideias para aplicação da pesquisa do GAL

	(1) Serie temporais -> para o rastreamento de amostras (estudo epidemiológico para o apoio do SUS)
	(2) Qual a linha de tendência do Zika no RJ? 
		- Separar doenças por mês, por estado, por região?
		- Separar por doenças (metadados da "coluna de observações", minerada pelo Marco)
		- Cálculo da taxa de confirmação (= total / confirmados)
	(3) Dados do Gal no surto de 2015 - 2018, aproximandamente 20,000 linhas
